\documentclass{beamer}
\usetheme{metropolis}           % Use metropolis theme
\usepackage{graphicx}

\title{Parallel BDDs}
\date{\today}
\author{Evan Bergeron, Kevin Zheng}
% \institute{CMU}
\begin{document}
\maketitle

\section{Introduction}

\begin{frame}
We implemented a parallel binary decision diagram library, focusing on spatial locality and cache coherence.
\end{frame}

\begin{frame}
Binary decision diagrams (BDDs) are directed graphs that represent boolean functions. Once constructed, these graphs provide constant time equivalence checking. Unfortunately, constructing these graphs can be costly.
\end{frame}

\begin{frame}{Main Data Structures}
A collection of lockfree hash tables that double as memory allocators and directed graphs.

A fine-grained-locking hash table with versioning, yielding a constant time delete_all operation.
\end{frame}

\section{Approach}

\begin{frame}{DFS}
Initial serial implementation DFS'd on the graph using the if-then-else normal form operation as described in [2].

We wanted to focus on getting BFS to work for high degrees of data parallelism.
\end{frame}

\begin{frame}
A lossy memoization cache is shared between workers to avoid duplicate work.

Additionally, the DAG and unique table are merged as described in [2] to reduce memory footprint.
\end{frame}

\section{The Quest for Spatial Locality}

\begin{frame}{Open Addressing}
Our initial unique table used separate chaining with linked lists in each bucket. After decided that we wanted to focus on memory locality, we switched a linear-probing, open-addressing scheme.

In a parallel context, this hash table must be able to be read and written to concurrently. Our final implementation is lockfree, heavily based on [11].
\end{frame}

\begin{frame}{Node Managers}
To improve spatial locality, we implemented **node managers**, as described in [3, 4].

Separate arrays are used for each variable id.
\end{frame}

\begin{frame}{A new struct definition}
Similar to [3], we inline the variable in the ``pointer'' we pass around by value. This avoids unnecassary memory dereferences.
\end{frame}

\begin{frame}{A new struct definition}
\begin{verbatim}
struct bdd_ptr_packed {
  uint16_t varid;
  uint32_t idx;
} __attribute__((packed));

struct bdd {
  bdd_ptr_packed lo;    // 6 bytes
  bdd_ptr_packed hi;    // 6 bytes
  uint16_t varid;       // 2 bytes
  uint16_t refcount;    // 2 bytes
};
\end{verbatim}
\end{frame}

\begin{frame}{Caveat}
In our current setup, this struct definition prevents us from having a dynamically-sized node manager. An additional layer of indirection would need to be added to allow this, something we did not get around to implementing.
\end{frame}

\begin{frame}{It's pretty small, though}
16bytes, which lets us we a compare and swap.
\end{frame}

\section{BFS}

\begin{frame}
We have an expand and reduce phase. We use an ITE formulation of BFS, similar to [8].

Lets us reuse code from DFS and if need be, switch between the two.
\end{frame}

\begin{frame}{We reuse the expand and reduce queue}
We only saw this approach in a couple of papers. A lot of the literature has thread-local expand queues and a global reduce queue.

Because of this reuse, the expand queue is shared between workers.
\end{frame}

\section{The Request Table}

\begin{frame}
Maps ite triples to a result node. Using the ite formulation forces the keys to be pretty large, forcing us to use fine-grained-locking in leu of compare and swaps.
\end{frame}

\begin{frame}{Reuse across calls}
This is what saved us. Each BFS call needs a fresh request table. It's far too expensive to reallocate the table each time.

We introduce a version counter. To delete_all, simply increment the version by one. Table entries with obsolete versions are treated as empty.
\end{frame}

\begin{frame}{Computer Nerd Details - XO Laptop Touch (2013)\cite{faq}}
\begin{itemize}
\item 1GB - 2GB RAM
\item 8GB SSD storage
\item Keyboard, touchpad
\item ARM processor
\item 5 GHz wifi support
\item Bluetooth
\item HDMI
\item Accelerometer
\end{itemize}
\end{frame}

\begin{frame}{Computer Nerd Details - XO Laptop Touch (2013)}
\begin{itemize}
\item 1200 x 900 resolution
\item 7.5`` screen
\item Lots of integrated peripherals
\end{itemize}
\end{frame}

\begin{frame}{Computer Nerd Details - Software}
\begin{itemize}
\item Fedora-based operating system (students can have root)
\item Custom-built Sugar GUI (written in python, GTK+) \cite{sugar}
\end{itemize}
\end{frame}

\begin{frame}{Computer Nerd Details - Summary}
Specs roughly equivalent to an iPhone. Strong emphasis on low power consumption. It uses commonly used open-source software.
\end{frame}

\section{Cost}

\begin{frame}[standout]
  The laptop costs around \$200 USD all told
\end{frame}

\begin{frame}[standout]
  Most low-income countries spend around \$48 USD per primary student per year
\end{frame}

\begin{frame}
  The laptops are advertised as having a lifespan of around 4 years. This would place the laptop as the sole cost of education for a single child over the course of 4 years.
\end{frame}

\section{Effectiveness}

\begin{frame}{Peru}
A 15-month study across 319 primary schools in Peru found \alert{no evidence} of any effect on enrollment or math and language tests. \cite{peru}
\end{frame}

\begin{frame}{Peru - Background}
\begin{itemize}
\item Spends an average of \$438 per primary student per year
\item 902,000 laptops purchased
\item Though the laptops are equipped with wifi, the pilot schools in Peru did not have internet.
\item Most of these schools had low levels of access to water and proper sewage systems.
\end{itemize}
\end{frame}

\begin{frame}{Peru - Implementation}
\begin{itemize}
\item Though OLPC intends for students to take the laptops home, school policies placed the financial responsibility for damage on the student's family if damaged at home.
\item In the week before survey, only 40\% of students took laptops home.
\end{itemize}
\end{frame}

\begin{frame}{Peru - Conclusion}
``Intense access to computers does not lead to measurable effects in academic achievement, but it did generate some positive impact on general cognitive skills.'' %\cite{peru}

Authors suggest the importance of providing the technology with an associated pedagogical model to help guide the use of the devices.
\end{frame}

\begin{frame}{Uruguay - 2007}
Results from the first two years of the program's implementation had \alert{no effects} on math and reading scores.\cite{uru}

Possible explanations:
\begin{itemize}
\item No compulsory teacher training
\item Often used to search for information on internet
\end{itemize}
\end{frame}

\begin{frame}{Uruguay - 2007}
\begin{itemize}
\item Over 350,000 students involved
\item \$260 per child, with \$21 annual cost
\item 2013 study showed that \alert{only 4.1\%} of laptops were used ``all'' or ``most'' days in 2012.
\end{itemize}
\end{frame}

\begin{frame}{Falling Usage}
In Uruguay, only 21.5\% of teachers report using the XOs in class on a near daily basis. In Alabama, 80.3\% percent of students report them never or seldom use the laptops. \cite{mark}
\end{frame}

\begin{frame}{Lack of Integration}
Mark Warschauer of UC Irvine in particular argues that OLPC is an example of the pitfalls of technocratic worldviews and a lack of understanding of long term solutions. \cite{mark}

Many agree that more integrated approach would likely lead to better results. \cite{uru, peru}
\end{frame}

\begin{frame}
``The OLPC deployments that simply tried to hand out laptops have failed because they ignored local contexts.''\cite{mark}
\end{frame}

\section{Conclusion}

\begin{frame}{My Take}
I think the numbers speak for themselves. Low usage, no demonstrable evidence that the machines increase test scores, high cost, and lack of integration all point to OLPC being a poorly implemented ICTD project.

It doesn't seem like project leaders attempted to grasp the needs of their target audience prior to embarking on this investment.
\end{frame}

\begin{frame}{My Take}
Some of these schools don't have proper water and sewage systems, and we want to focus on giving \textit{every child} a laptop they can take home before fixing the water?
\end{frame}

\begin{frame}[allowframebreaks]{References}
\bibliography{presentation}
\bibliographystyle{abbrv}
\end{frame}

\end{document}
